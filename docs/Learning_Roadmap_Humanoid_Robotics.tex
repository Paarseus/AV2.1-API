\documentclass[11pt,letterpaper]{article}

% Packages
\usepackage[utf8]{inputenc}
\usepackage[T1]{fontenc}
\usepackage{geometry}
\usepackage{hyperref}
\usepackage{enumitem}
\usepackage{titlesec}
\usepackage{xcolor}
\usepackage{booktabs}
\usepackage{longtable}
\usepackage{fancyhdr}
\usepackage{graphicx}

% Page geometry
\geometry{margin=1in}

% Colors
\definecolor{tier1}{RGB}{180, 50, 50}
\definecolor{tier2}{RGB}{50, 120, 180}
\definecolor{tier3}{RGB}{80, 140, 80}
\definecolor{linkblue}{RGB}{0, 102, 204}

% Hyperref setup
\hypersetup{
    colorlinks=true,
    linkcolor=linkblue,
    urlcolor=linkblue,
    citecolor=linkblue
}

% Section formatting
\titleformat{\section}{\Large\bfseries}{\thesection}{1em}{}
\titleformat{\subsection}{\large\bfseries}{\thesubsection}{1em}{}
\titleformat{\subsubsection}{\normalsize\bfseries}{\thesubsubsection}{1em}{}

% Header/Footer
\pagestyle{fancy}
\fancyhf{}
\rhead{Learning Roadmap}
\lhead{AV to Humanoid Robotics}
\cfoot{\thepage}

\title{\textbf{Strategic Learning Roadmap}\\[0.5em]
\Large From Autonomous Vehicles to Humanoid Robotics}
\author{Autonomous Vehicle Laboratory\\Cal Poly Pomona}
\date{January 2026}

\begin{document}

\maketitle
\tableofcontents
\newpage

%==============================================================================
\section{Your Current Position}
%==============================================================================

Based on analyzing the UTM Navigator codebase, you've built production-grade autonomous vehicle software with strong foundations and identified gaps.

\subsection{Strong Foundations}

\begin{itemize}
    \item Vehicle kinematics \& coordinate systems (WGS84/UTM/Vehicle frame transforms)
    \item Path tracking (Pure Pursuit with adaptive lookahead)
    \item Reactive local planning (DWA with Ackermann constraints)
    \item Grid-based mapping (log-odds Bayesian updates, Bresenham raycasting)
    \item Real hardware integration (Xsens RTK-GPS, Velodyne LIDAR, Teensy CAN)
    \item Production software practices (threading, config management, modular architecture)
\end{itemize}

\subsection{Key Gaps Identified}

\begin{itemize}
    \item No state estimation (Kalman filters, sensor fusion)
    \item No trajectory optimization (MPC, LQR)
    \item No reinforcement learning or sim2real
    \item Vision/deep learning exists but disconnected from control loop
    \item No dexterous manipulation experience
\end{itemize}

%==============================================================================
\section{The LLM Era Reality Check}
%==============================================================================

The landscape has changed. Here's what LLMs \textbf{can} and \textbf{cannot} do for robotics engineers in 2026.

\subsection{LLMs CAN Handle (Delegate Freely)}

\begin{itemize}
    \item Boilerplate code generation
    \item API integrations and driver code
    \item Documentation and technical writing
    \item Debugging known patterns
    \item Code refactoring and cleanup
    \item Basic algorithm implementations (PID, A*, etc.)
\end{itemize}

\subsection{LLMs CANNOT Replace (Must Master Deeply)}

\begin{itemize}
    \item \textbf{Physical intuition} -- understanding why a robot falls, drifts, or fails
    \item \textbf{System debugging} -- tracing issues across sensors $\rightarrow$ perception $\rightarrow$ planning $\rightarrow$ control $\rightarrow$ hardware
    \item \textbf{Dynamics modeling} -- choosing the right simplifications for real-time control
    \item \textbf{Reward engineering} -- designing cost functions that produce desired emergent behavior
    \item \textbf{Sim2real transfer} -- understanding what physics matter and what can be randomized
    \item \textbf{Safety reasoning} -- knowing failure modes and designing fallbacks
\end{itemize}

\textbf{Key insight}: LLMs are tools, not replacements. The engineers who thrive will be those who can specify problems precisely, validate solutions physically, and debug across the full stack.

%==============================================================================
\section{Mathematical Foundations}
%==============================================================================

Before tackling the robotics topics that follow (State Estimation, MPC, RL, etc.), you need solid mathematical foundations. The robotics algorithms are applications of math---without the fundamentals, you'll be pattern-matching code snippets without understanding why things work or how to debug when they don't.

\subsection{Depth Levels Defined}

Not all math needs the same depth. Here's what each level means:

\begin{description}
    \item[\textcolor{tier3}{Conceptual Understanding}] Know what the technique does and when to use it. Can use library implementations effectively. Sufficient for supporting skills.
    \item[\textcolor{tier2}{Computational Fluency}] Can compute by hand and implement from scratch. Understand the derivations. Can debug numerical issues. Required for core skills.
    \item[\textcolor{tier1}{Proof-Level Mastery}] Understand why theorems hold and their conditions. Can modify techniques for novel situations. Required for deep specialization.
\end{description}

%------------------------------------------------------------------------------
\subsection{Core Mathematical Areas}
%------------------------------------------------------------------------------

%..............................................................................
\subsubsection{Linear Algebra}

\textbf{Depth:} \textcolor{tier1}{Proof-Level} for fundamentals, \textcolor{tier2}{Computational} for decompositions

\textbf{Why it matters:} Linear algebra is the language of robotics. State estimation is matrix algebra. MPC solves matrix equations. Transformations are matrices. If you're weak here, everything else becomes harder.

\textbf{Topics to master:}
\begin{itemize}
    \item Matrix operations, vector spaces, linear independence
    \item Eigenvalues/eigenvectors and their geometric meaning
    \item Singular Value Decomposition (SVD) -- appears everywhere
    \item Matrix decompositions: QR, LU, Cholesky
    \item Positive definite matrices (critical for optimization)
    \item Matrix calculus: gradients, Jacobians w.r.t. matrices
\end{itemize}

\textbf{Enables:} State Estimation, MPC, Computer Vision, Dynamics, Everything

\textbf{Resources:}
\begin{itemize}
    \item Course: MIT 18.06 Linear Algebra (Gilbert Strang) -- FREE
    \item Book: \textit{Linear Algebra Done Right} by Axler (theory focus)
    \item Book: \textit{Matrix Computations} by Golub \& Van Loan (numerical focus)
    \item Video: 3Blue1Brown ``Essence of Linear Algebra'' (visual intuition)
\end{itemize}

\textbf{Estimated time:} 4--6 weeks

%..............................................................................
\subsubsection{Multivariable Calculus}

\textbf{Depth:} \textcolor{tier2}{Computational Fluency}

\textbf{Why it matters:} Kinematics is built on Jacobians. Optimization uses gradients and Hessians. Control systems linearize around equilibria. You'll compute derivatives constantly.

\textbf{Topics to master:}
\begin{itemize}
    \item Gradients, directional derivatives
    \item Jacobian matrices (input-output relationships)
    \item Hessian matrices (curvature for optimization)
    \item Chain rule for composed functions
    \item Taylor series expansions (linearization)
    \item Implicit function theorem
\end{itemize}

\textbf{Enables:} Control, Optimization, Kinematics, Machine Learning

\textbf{Resources:}
\begin{itemize}
    \item Course: MIT 18.02 Multivariable Calculus -- FREE
    \item Book: \textit{Calculus on Manifolds} by Spivak (rigorous)
    \item Video: 3Blue1Brown ``Essence of Calculus''
\end{itemize}

\textbf{Estimated time:} 3--4 weeks

%..............................................................................
\subsubsection{Differential Equations}

\textbf{Depth:} \textcolor{tier2}{Computational Fluency}

\textbf{Why it matters:} Robot dynamics are differential equations. Control theory is about shaping their solutions. Simulation requires numerical integration. This is the math of motion.

\textbf{Topics to master:}
\begin{itemize}
    \item First-order ODEs (analytical and numerical solutions)
    \item Linear systems of ODEs (matrix exponential)
    \item Stability analysis (eigenvalue conditions)
    \item Numerical integration: Euler, Runge-Kutta (RK4)
    \item State-space representation
    \item Phase portraits and equilibrium analysis
\end{itemize}

\textbf{Enables:} Dynamics, Control, Simulation, State Estimation

\textbf{Resources:}
\begin{itemize}
    \item Course: MIT 18.03 Differential Equations -- FREE
    \item Book: \textit{Nonlinear Dynamics and Chaos} by Strogatz (excellent intuition)
    \item Library: \texttt{scipy.integrate} for numerical methods
\end{itemize}

\textbf{Estimated time:} 4--5 weeks

%..............................................................................
\subsubsection{Probability \& Statistics}

\textbf{Depth:} \textcolor{tier1}{Proof-Level} for Gaussian/Bayesian concepts, \textcolor{tier2}{Computational} for estimation

\textbf{Why it matters:} Sensors are noisy. State estimation is probabilistic inference. RL is built on expectation and variance. You need to think in distributions, not point values.

\textbf{Topics to master:}
\begin{itemize}
    \item Bayes' theorem and Bayesian inference
    \item Gaussian (Normal) distributions
    \item Multivariate Gaussians, covariance matrices
    \item Maximum Likelihood Estimation (MLE)
    \item Maximum A Posteriori (MAP) estimation
    \item Markov chains and Markov property
    \item Conditional independence
\end{itemize}

\textbf{Enables:} State Estimation, SLAM, Reinforcement Learning, Perception

\textbf{Resources:}
\begin{itemize}
    \item Course: MIT 6.041 Probabilistic Systems Analysis -- FREE
    \item Book: \textit{Probabilistic Robotics} by Thrun (robotics-specific)
    \item Book: \textit{Pattern Recognition and Machine Learning} by Bishop (thorough)
\end{itemize}

\textbf{Estimated time:} 5--6 weeks

%..............................................................................
\subsubsection{3D Geometry \& Spatial Transformations}

\textbf{Depth:} \textcolor{tier1}{Proof-Level} (this is core robotics math)

\textbf{Why it matters:} Robots move in 3D space. Every joint, every camera, every sensor has a pose. Representing and composing rotations correctly is non-trivial. Mess this up and your robot will move wrong.

\textbf{Topics to master:}
\begin{itemize}
    \item Rotation matrices and SO(3) (the rotation group)
    \item Euler angles and their limitations (gimbal lock)
    \item Quaternions (singularity-free rotation representation)
    \item Homogeneous coordinates and SE(3) (rigid body transforms)
    \item Transformation composition and inverses
    \item Introduction to Lie groups and Lie algebras (so(3), se(3))
\end{itemize}

\textbf{Enables:} Kinematics, SLAM, Computer Vision, Control

\textbf{Resources:}
\begin{itemize}
    \item Book: \textit{Modern Robotics} by Lynch \& Park (excellent treatment)
    \item Book: \textit{A Mathematical Introduction to Robotic Manipulation} by Murray, Li, Sastry
    \item Library: \texttt{scipy.spatial.transform}, Pinocchio
\end{itemize}

\textbf{Estimated time:} 4--5 weeks

%..............................................................................
\subsubsection{Optimization Theory}

\textbf{Depth:} \textcolor{tier2}{Computational Fluency}

\textbf{Why it matters:} MPC is optimization. Motion planning is optimization. State estimation (batch form) is optimization. Machine learning is optimization. This skill pays dividends everywhere.

\textbf{Topics to master:}
\begin{itemize}
    \item Convex sets and convex functions
    \item Gradient descent and convergence conditions
    \item Newton's method and quadratic convergence
    \item Lagrange multipliers (equality constraints)
    \item KKT conditions (inequality constraints)
    \item Quadratic Programming (QP)
    \item Basics of nonlinear programming
\end{itemize}

\textbf{Enables:} MPC, Motion Planning, State Estimation, Machine Learning

\textbf{Resources:}
\begin{itemize}
    \item Course: Stanford EE364a Convex Optimization (Boyd) -- FREE
    \item Book: \textit{Convex Optimization} by Boyd \& Vandenberghe -- FREE online
    \item Library: CVXPY, CasADi, OSQP
\end{itemize}

\textbf{Estimated time:} 5--6 weeks

%------------------------------------------------------------------------------
\subsection{Mathematics to Robotics Mapping}
%------------------------------------------------------------------------------

This table shows how each mathematical area connects to robotics applications:

\begin{longtable}{@{}p{3cm}p{5cm}p{5.5cm}@{}}
\toprule
\textbf{Math Area} & \textbf{Key Concepts} & \textbf{Robotics Applications} \\
\midrule
\endhead
Linear Algebra & SVD, Cholesky, eigendecomposition & Kalman filter, MPC solving, least squares \\
\addlinespace
Multivariable Calculus & Jacobians, Hessians, chain rule & Kinematics, optimization, linearization \\
\addlinespace
Differential Equations & Linear systems, stability, RK4 & Dynamics simulation, control design \\
\addlinespace
Probability & Bayes, Gaussians, covariance & State estimation, SLAM, sensor fusion \\
\addlinespace
3D Geometry & SO(3), SE(3), quaternions & Transforms, SLAM, computer vision \\
\addlinespace
Optimization & QP, KKT conditions, gradients & MPC, trajectory optimization, learning \\
\bottomrule
\end{longtable}

%------------------------------------------------------------------------------
\subsection{Recommended Math Learning Sequence}
%------------------------------------------------------------------------------

The dependencies between topics suggest this order:

\begin{enumerate}
    \item \textbf{Linear Algebra} (4--6 weeks) -- Foundation for everything else
    \item \textbf{Multivariable Calculus} (3--4 weeks) -- Requires linear algebra
    \item \textbf{Probability \& Statistics} (5--6 weeks) -- Requires linear algebra; can parallel with \#4
    \item \textbf{Differential Equations} (4--5 weeks) -- Requires calculus
    \item \textbf{3D Geometry} (4--5 weeks) -- Requires linear algebra; can start after \#1
    \item \textbf{Optimization} (5--6 weeks) -- Requires calculus + linear algebra; best done last
\end{enumerate}

\textbf{Total time estimates:}
\begin{itemize}
    \item Full sequential path: 25--32 weeks (6--8 months)
    \item Accelerated path (with undergraduate background): 12--16 weeks
    \item Parallel study (if disciplined): 16--20 weeks
\end{itemize}

%------------------------------------------------------------------------------
\subsection{How Math Flows into Robotics Skills}
%------------------------------------------------------------------------------

\begin{verbatim}
      Linear Algebra          Multivariable Calculus
            |                         |
            v                         v
    +-------+-------+         +-------+-------+
    |               |         |               |
    v               v         v               v
Probability    3D Geometry   Diff Eqs    Optimization
    |               |           |             |
    +-------+-------+-----------+-------------+
            |               |
            v               v
    State Estimation    Trajectory Opt (MPC)
            |               |
            +-------+-------+
                    |
                    v
         Reinforcement Learning
                    |
                    v
           Legged Locomotion
\end{verbatim}

\textbf{Key insight}: If you skip the math foundations, you'll hit walls when debugging algorithms that ``should work'' but don't. The math gives you the ability to reason about \textit{why} things fail.

%==============================================================================
\section{The Learning Roadmap}
%==============================================================================

%------------------------------------------------------------------------------
\subsection{\textcolor{tier1}{TIER 1: FOUNDATIONAL (Deep Study Required)}}
%------------------------------------------------------------------------------

These are the ``non-negotiable'' skills that everything else builds on. You cannot shortcut these.

%..............................................................................
\subsubsection{1. State Estimation \& Sensor Fusion}

\textbf{Depth: DEEP (3--4 months focused study)}

\textbf{Why it matters:} Every legged robot needs to know where it is and how fast it's moving. Your codebase has zero state estimation---you're reading raw GPS. Real systems fuse GPS, IMU, wheel encoders, and visual odometry.

\textbf{What to learn:}
\begin{itemize}
    \item Kalman Filter fundamentals (linear systems)
    \item Extended Kalman Filter (EKF) for nonlinear systems
    \item Unscented Kalman Filter (UKF) -- better for highly nonlinear
    \item Factor graphs and batch optimization (GTSAM)
    \item IMU preintegration
\end{itemize}

\textbf{Practical project:} Add an EKF to your AV platform that fuses GPS + IMU. You'll immediately see smoother position estimates and can detect GPS dropouts.

\textbf{Resources:}
\begin{itemize}
    \item Book: \textit{Probabilistic Robotics} by Thrun, Burgard, Fox (the bible)
    \item Book: \textit{State Estimation for Robotics} by Tim Barfoot (more modern)
    \item Course: Coursera ``State Estimation and Localization for Self-Driving Cars''
    \item Library: FilterPy (Python), GTSAM (C++/Python)
\end{itemize}

\textbf{Connection to humanoids:} Legged robots have no wheel encoders. They must fuse IMU + kinematics + contact detection. Same math, harder problem.

%..............................................................................
\subsubsection{2. Trajectory Optimization \& MPC}

\textbf{Depth: DEEP (3--4 months focused study)}

\textbf{Why it matters:} Your Pure Pursuit + PID is reactive---it sees the path and follows. MPC is predictive---it plans trajectories that respect constraints (motor limits, stability, obstacles) over a horizon.

\textbf{What to learn:}
\begin{itemize}
    \item Linear MPC formulation
    \item Nonlinear MPC (harder but necessary for complex dynamics)
    \item Quadratic Programs (QP) and solvers (OSQP, ECOS)
    \item iLQR (iterative Linear Quadratic Regulator)
    \item Whole-body control for humanoids
\end{itemize}

\textbf{Practical project:} Replace your Pure Pursuit with MPC for path tracking. You'll see better performance in sharp turns and can add speed-dependent constraints.

\textbf{Resources:}
\begin{itemize}
    \item Course: MIT 6.832 ``Underactuated Robotics'' (Russ Tedrake) -- FREE
    \item Book: \textit{Predictive Control for Linear and Hybrid Systems} (Borrelli)
    \item Library: CasADi, ACADOS, Drake, Crocoddyl
    \item Tutorial: Machines in Motion Lab MPC resources
\end{itemize}

\textbf{Connection to humanoids:} MPC is how Atlas, Digit, and Optimus maintain balance. They're solving optimization problems at 100Hz+ to generate joint torques.

%..............................................................................
\subsubsection{3. Reinforcement Learning \& Sim2Real}

\textbf{Depth: DEEP (4--6 months)}

\textbf{Why it matters:} This is how legged locomotion is solved now. Every humanoid company (Figure, Agility, Tesla, Boston Dynamics) uses RL trained in simulation, transferred to real hardware.

\textbf{What to learn:}
\begin{itemize}
    \item RL fundamentals (MDPs, policy gradients, value functions)
    \item PPO (Proximal Policy Optimization) -- the workhorse algorithm
    \item Domain randomization for sim2real
    \item Reward shaping and curriculum learning
    \item Sim2sim validation
\end{itemize}

\textbf{Practical project:} Train a simulated quadruped (Unitree Go1) to walk using Isaac Gym or MuJoCo. This is the on-ramp to humanoid RL.

\textbf{Resources:}
\begin{itemize}
    \item Course: UC Berkeley CS 285 ``Deep RL'' (Sergey Levine) -- FREE
    \item Framework: Isaac Gym / Isaac Lab (NVIDIA) or MuJoCo + Brax
    \item Code: \url{https://github.com/roboterax/humanoid-gym}
    \item Code: \url{https://github.com/unitreerobotics/unitree_rl_gym}
    \item Paper: ``Learning Agile Locomotion'' (ETH Zurich)
\end{itemize}

\textbf{Connection to humanoids:} This is THE skill for humanoid locomotion. Companies literally hire based on Isaac Gym experience.

%------------------------------------------------------------------------------
\subsection{\textcolor{tier2}{TIER 2: HIGH-LEVERAGE (Solid Understanding Required)}}
%------------------------------------------------------------------------------

These amplify your Tier 1 skills and are directly applicable to humanoid work.

%..............................................................................
\subsubsection{4. Rigid Body Dynamics \& Contact Mechanics}

\textbf{Depth: MEDIUM-DEEP (2--3 months)}

\textbf{Why it matters:} Humanoids are complex multi-body systems. You need to understand Lagrangian/Hamiltonian mechanics, constraint forces, and how contact works.

\textbf{What to learn:}
\begin{itemize}
    \item Spatial vector algebra (Featherstone notation)
    \item Forward/inverse dynamics
    \item Contact modeling (compliant vs rigid)
    \item Friction cones and contact wrenches
\end{itemize}

\textbf{Resources:}
\begin{itemize}
    \item Book: \textit{Rigid Body Dynamics Algorithms} by Roy Featherstone
    \item Library: Pinocchio (fast dynamics library)
    \item MIT 6.832 covers this well
\end{itemize}

\textbf{Connection to your work:} Your bicycle model is a start, but humanoids have 30+ DOF with closed kinematic chains.

%..............................................................................
\subsubsection{5. Computer Vision for Robotics}

\textbf{Depth: MEDIUM (2--3 months)}

\textbf{Why it matters:} You have YOLOPv2 in your codebase but it's not integrated. Humanoids need vision for manipulation, navigation, and human interaction.

\textbf{What to learn:}
\begin{itemize}
    \item Visual odometry / Visual SLAM
    \item Depth estimation and 3D reconstruction
    \item Object detection and pose estimation
    \item Foundation models (CLIP, SAM, DINO)
\end{itemize}

\textbf{Resources:}
\begin{itemize}
    \item Course: University of Bonn ``Photogrammetry'' (Cyrill Stachniss)
    \item Library: OpenCV, ORB-SLAM3, DROID-SLAM
    \item Modern: Use foundation models via API when possible
\end{itemize}

\textbf{Connection to humanoids:} Figure AI uses vision extensively for manipulation. Not as critical as locomotion skills but important.

%..............................................................................
\subsubsection{6. Real-Time Systems}

\textbf{Depth: MEDIUM (1--2 months)}

\textbf{Why it matters:} Your Python code runs at ``soft real-time'' (best effort). Production humanoids need hard guarantees---1kHz control loops that never miss a deadline.

\textbf{What to learn:}
\begin{itemize}
    \item PREEMPT\_RT Linux
    \item Real-time priorities and scheduling
    \item Lock-free programming patterns
    \item ROS2 real-time considerations
\end{itemize}

\textbf{Resources:}
\begin{itemize}
    \item Book: \textit{Real-Time Systems} by Jane Liu
    \item Practical: Run your control loop at 1kHz on PREEMPT\_RT Linux
\end{itemize}

\textbf{Connection to your work:} Your 20Hz control loop is fine for slow AV navigation. Humanoid balance requires 500Hz--1kHz.

%------------------------------------------------------------------------------
\subsection{\textcolor{tier3}{TIER 3: NICE-TO-HAVE (Learn As Needed)}}
%------------------------------------------------------------------------------

These are valuable but can be learned on-demand or delegated to LLMs for assistance.

\subsubsection{7. ROS2 \& Robot Middleware}
\textbf{Depth: LIGHT (learn when needed)} \\
Industry standard, but you can learn it when you join a team that uses it. Your current modular architecture is actually cleaner for research.

\subsubsection{8. CAD \& Mechanical Design}
\textbf{Depth: LIGHT} \\
As a software/controls person, you'll collaborate with mechanical engineers. Basic Fusion360/SolidWorks literacy helps communication.

\subsubsection{9. PCB Design \& Embedded Systems}
\textbf{Depth: LIGHT (unless specializing)} \\
Your Teensy integration is solid. Deeper embedded work is a separate specialization.

\subsubsection{10. Safety Systems \& Standards}
\textbf{Depth: LEARN ON JOB} \\
ISO 13482 (service robots), functional safety---important but company-specific.

%==============================================================================
\section{Manipulation: The Other Half of Humanoids}
%==============================================================================

Locomotion is only half the problem. Manipulation (dexterous hands, grasping, tool use) is equally important but uses different techniques.

\textbf{Key skills:}
\begin{itemize}
    \item Grasp planning and contact reasoning
    \item Imitation learning from human demonstrations
    \item Tactile sensing and force control
    \item Bimanual coordination
\end{itemize}

\textbf{Recommended approach:} Focus on locomotion first (Tier 1), then pivot to manipulation. The RL skills transfer directly.

%==============================================================================
\section{Suggested Timeline (24--32 months)}
%==============================================================================

This timeline includes math foundations as a prerequisite phase. If you already have strong undergraduate-level math, you can compress or skip the first phase.

\begin{table}[h]
\centering
\begin{tabular}{@{}lll@{}}
\toprule
\textbf{Months} & \textbf{Focus Area} & \textbf{Practical Application} \\
\midrule
\multicolumn{3}{l}{\textit{Phase 0: Mathematical Foundations}} \\
\addlinespace
$-$6 to $-$4 & Linear Algebra + Calculus & MIT 18.06, 18.02, implement from scratch \\
$-$4 to $-$2 & Probability + Diff Eqs & Bayesian inference, numerical integration \\
$-$2 to 0 & 3D Geometry + Optimization & SE(3) transforms, convex optimization \\
\addlinespace
\multicolumn{3}{l}{\textit{Phase 1: Robotics Core}} \\
\addlinespace
1--4 & State Estimation (EKF/UKF) & Add sensor fusion to your AV \\
4--8 & Trajectory Optimization (MPC) & Replace Pure Pursuit with MPC \\
8--14 & Reinforcement Learning \& Sim2Real & Train quadruped in Isaac Gym \\
14--18 & Dynamics + Real-time & Pinocchio, PREEMPT\_RT \\
18--24 & First humanoid project & Unitree H1 simulation or custom build \\
\bottomrule
\end{tabular}
\end{table}

\textbf{Note:} The ``negative months'' in Phase 0 indicate this work should be done \textit{before} starting the robotics curriculum. If you're strong in math, start at Month 1. If you need the full math foundation, add 4--6 months to the timeline.

%==============================================================================
\section{Recommended Hardware Path}
%==============================================================================

\begin{enumerate}
    \item \textbf{Keep your AV platform} -- It's a great testbed for state estimation and MPC
    \item \textbf{Get a Unitree Go1/Go2} (\$1,600--2,700) -- Industry standard for learning legged locomotion
    \item \textbf{Unitree H1} (when ready) -- Most accessible humanoid platform
    \item \textbf{Or build custom} -- Many labs build small bipeds for research
\end{enumerate}

%==============================================================================
\section{How These Skills Connect}
%==============================================================================

\begin{verbatim}
State Estimation <---> Trajectory Optimization
       |                        |
       +-----> Reinforcement Learning <----+
                      |
            Sim2Real Transfer
                      |
             Legged Locomotion
                      |
             Humanoid Control
                      |
               Manipulation
\end{verbatim}

Every Tier 1 skill feeds into the others:
\begin{itemize}
    \item State estimation gives you belief state for MPC
    \item MPC teaches you constraint handling for reward design
    \item RL policies need dynamics models for sim2real
\end{itemize}

%==============================================================================
\section{Final Thoughts}
%==============================================================================

\subsection{What Makes You Competitive}

\begin{enumerate}
    \item You've shipped real hardware (rare among students)
    \item Your software architecture is production-grade
    \item You understand the full sense-plan-act pipeline
\end{enumerate}

\subsection{What You Need}

\begin{enumerate}
    \item Mathematical depth in optimization and estimation
    \item RL/sim2real for modern locomotion
    \item Experience with legged platforms
\end{enumerate}

\subsection{Career Positioning}

Companies like Figure AI, Tesla Optimus, Agility, and Apptronik hire from:
\begin{itemize}
    \item Boston Dynamics alumni
    \item ETH Zurich / MIT / CMU robotics labs
    \item Self-taught engineers with Isaac Gym experience and real robot demos
\end{itemize}

\textbf{Your path:} Build a portfolio of simulated + real legged robot demos. A YouTube video of a quadruped you trained from scratch in Isaac Gym is worth more than most resumes.

%==============================================================================
\section{References}
%==============================================================================

\begin{enumerate}
    \item 10 Must-Have Skills for Robotics Engineers 2026: \url{https://www.edstellar.com/blog/robotics-engineers-skills}
    \item Learning-based Legged Locomotion (IJRR): \url{https://journals.sagepub.com/doi/10.1177/02783649241312698}
    \item Humanoid-Gym: Zero-Shot Sim2Real: \url{https://github.com/roboterax/humanoid-gym}
    \item State Estimation for Robotics: \url{https://www.numberanalytics.com/blog/state-estimation-in-robotics-ultimate-guide}
    \item MPC vs PID for High-Speed Robots: \url{https://eureka.patsnap.com/article/pid-vs-model-predictive-control-mpc-which-is-better-for-high-speed-robots}
    \item LLMs in Robotics (Georgia Tech): \url{https://github.com/GT-RIPL/Awesome-LLM-Robotics}
    \item Figure vs Tesla Humanoid Analysis: \url{https://www.diamandis.com/blog/abundance-43-figure-vs-tesla}
    \item Dexterous Manipulation Survey: \url{https://www.frontiersin.org/journals/robotics-and-ai/articles/10.3389/frobt.2025.1682437/full}
\end{enumerate}

\end{document}
