\documentclass[conference]{IEEEtran}

% Packages
\usepackage{cite}
\usepackage{amsmath,amssymb,amsfonts}
\usepackage{graphicx}
\usepackage{textcomp}
\usepackage{xcolor}
\usepackage{booktabs}
\usepackage{hyperref}
\usepackage{balance}
\usepackage{tikz}
\usetikzlibrary{shapes.geometric, arrows.meta, positioning, fit, backgrounds}
\usepackage{listings}
\lstset{
    language=Python,
    basicstyle=\footnotesize\ttfamily,
    keywordstyle=\color{blue},
    commentstyle=\color{gray},
    numbers=none,
    frame=single,
    breaklines=true,
    captionpos=b
}

\begin{document}

\title{Vibration-Based Road Surface Classification for Low-Speed Autonomous Campus Vehicles: An Educational Platform Approach}

\author{
\IEEEauthorblockN{Mohammadparsa Ghasemi, Behnam Bahr, and Viviane Seyranian}
\IEEEauthorblockA{California State Polytechnic University, Pomona\\
Pomona, CA, USA}
}

\maketitle

%==============================================================================
\begin{abstract}
Autonomous vehicles operating in campus and urban environments require real-time awareness of road surface conditions to optimize ride comfort, energy efficiency, and safety. While dedicated vibration sensors and smartphone-based solutions have demonstrated road surface classification capabilities, these approaches either add cost or compromise measurement quality. This paper presents a dual-purpose sensing approach that leverages a high-quality GNSS/INS unit---the Xsens MTi-680G, already present on autonomous vehicles for navigation---for simultaneous road surface classification. Using the 100~Hz tri-axial accelerometer data from this navigation-grade IMU, we extract time-domain features (RMS acceleration, peak-to-peak amplitude, crest factor, kurtosis) and frequency-domain features (spectral centroid, band power ratios) to train machine learning classifiers. We evaluate both traditional methods (Random Forest, SVM) and deep learning approaches (1D-CNN, CNN-LSTM) on a dataset collected across five road surface types on a university campus: smooth asphalt, rough asphalt, speed bumps, potholes/patches, and concrete. Our CNN-LSTM architecture achieves [XX]\% classification accuracy with inference latency under 50~ms, enabling real-time road condition awareness. Beyond the technical contribution, we demonstrate how this research integrates into an undergraduate autonomous vehicle curriculum, providing students hands-on experience in signal processing, machine learning, and embedded systems. The platform, built on an electric utility vehicle converted to drive-by-wire operation, enables experiential learning while producing research-quality results. This work demonstrates that navigation-grade IMUs can serve dual purposes in autonomous vehicles, reducing system complexity while enabling vibration-based environmental perception.

% TODO: Fill in accuracy results after experiments
\end{abstract}

\begin{IEEEkeywords}
road surface classification, vibration analysis, autonomous vehicles, IMU, machine learning, signal processing, educational platform
\end{IEEEkeywords}

%==============================================================================
\section{Introduction}

The rapid deployment of autonomous vehicles in campus environments, last-mile delivery, and urban mobility applications creates new requirements for environmental perception beyond traditional obstacle detection. Road surface conditions directly impact vehicle dynamics, passenger comfort, energy consumption, and safety margins. A vehicle aware of upcoming rough pavement can proactively adjust speed, modify suspension settings (if available), or select alternative routes.

Current approaches to road surface monitoring fall into three categories. First, infrastructure-based systems embed sensors in roadways but require significant capital investment and maintenance~\cite{rahman2024}. Second, dedicated vehicle-mounted vibration sensors provide high-quality measurements but add cost and complexity to autonomous vehicle platforms~\cite{souza2025}. Third, smartphone-based crowdsourcing leverages ubiquitous accelerometers but suffers from inconsistent mounting, variable device quality, and limited sampling rates~\cite{vibration2021}.

This paper proposes a fourth approach: dual-purpose use of navigation-grade Inertial Measurement Units (IMUs) already present on autonomous vehicles. Modern GNSS/INS systems like the Xsens MTi-680G include high-quality tri-axial accelerometers operating at 100~Hz or higher---far exceeding the requirements for road surface characterization, where relevant frequencies typically fall below 50~Hz. By extracting vibration features from navigation IMU data, we enable road surface classification without additional hardware.

Our contributions are threefold:
\begin{enumerate}
    \item We demonstrate that navigation-grade IMUs can classify road surfaces with accuracy comparable to dedicated vibration sensing systems
    \item We present a complete feature extraction and classification pipeline optimized for real-time operation on autonomous vehicle compute platforms
    \item We describe curriculum integration enabling undergraduate students to gain hands-on experience in signal processing and machine learning through meaningful research contributions
\end{enumerate}

The remainder of this paper is organized as follows. Section~II reviews related work in vibration-based road classification. Section~III describes our autonomous vehicle platform and sensor specifications. Section~IV presents our methodology including feature extraction and classification approaches. Section~V details experimental setup and data collection. Section~VI presents results and analysis. Section~VII discusses educational integration. Section~VIII concludes with limitations and future work.

%==============================================================================
\section{Related Work}

\subsection{Vibration-Based Road Surface Classification}

Road surface classification from vehicle vibrations has attracted significant research attention. Early work focused on International Roughness Index (IRI) estimation using dedicated accelerometers~\cite{vibration2021}. The IRI, defined as accumulated suspension motion per unit distance traveled, serves as the standard metric for pavement condition assessment.

Recent advances in machine learning have enabled more nuanced classification beyond simple roughness estimation. Souza et al.~\cite{souza2025} demonstrated CNN-LSTM architectures for classifying five road surface types using multi-IMU configurations mounted on both sprung (vehicle body) and unsprung (wheel) masses. Their approach achieved 93.4\% accuracy but required dedicated sensor installations.

Smartphone-based approaches leverage the accelerometers in consumer devices. While accessible, these methods face challenges including variable mounting orientations, inconsistent sampling rates across devices, and interference from user interactions~\cite{pavement2023}. Additionally, smartphone accelerometers typically operate at lower quality levels than navigation-grade IMUs.

\subsection{Feature Extraction for Vibration Classification}

Vibration signals contain information in both time and frequency domains. Time-domain features include statistical measures such as root-mean-square (RMS) acceleration, peak-to-peak amplitude, crest factor (ratio of peak to RMS), kurtosis, and skewness. These features capture signal energy and distribution characteristics that distinguish smooth from rough surfaces.

Frequency-domain features reveal periodic components related to road texture. Spectral centroid indicates the ``center of mass'' of the frequency spectrum. Band power ratios---comparing energy in low (0--5~Hz), mid (5--15~Hz), and high (15--50~Hz) frequency bands---correlate with different surface characteristics. Low frequencies correspond to long-wavelength roughness, while higher frequencies indicate fine texture or discrete events like potholes.

Recent work explores time-frequency representations including short-time Fourier transforms (spectrograms) and wavelet decompositions. These approaches capture how frequency content evolves over time, potentially improving detection of transient events like speed bumps~\cite{pavement2024}.

\subsection{Deep Learning Approaches}

Convolutional neural networks (CNNs) have demonstrated strong performance on vibration classification tasks. 1D-CNNs process raw accelerometer time series, automatically learning relevant features without manual engineering~\cite{souza2025}. Adding Long Short-Term Memory (LSTM) layers enables modeling of temporal dependencies across consecutive windows.

A key consideration is the trade-off between classification accuracy and computational requirements. While deep learning approaches often achieve higher accuracy, they require more computational resources than traditional machine learning methods like Random Forests. For real-time autonomous vehicle applications, inference latency must remain below perception system cycle times---typically 50--100~ms.

%==============================================================================
\section{Platform Description}

\subsection{DRIVE Autonomous Vehicle Platform}

The Development Research Infrastructure for Vehicle Education (DRIVE) platform is an electric utility vehicle converted to full drive-by-wire operation at California State Polytechnic University, Pomona. The base vehicle is a 350-pound electric youth UTV powered by a 48V lithium battery system driving a 1.8~kW motor, with a 1.23-meter wheelbase and maximum speed electronically limited to 27~mph.

The vehicle features custom-designed electromechanical systems for steering (stepper motor with shaft encoder), throttle (electronic control with wheel speed encoder), and braking (linear actuator integrated with hydraulic system). A distributed Controller Area Network (CAN) architecture with four Teensy 4.1 microcontrollers implements closed-loop control on all actuators. The complete drive-by-wire conversion cost approximately \$12,000--15,000, making it accessible for educational institutions.

\subsection{Xsens MTi-680G GNSS/INS}

The primary navigation sensor is an Xsens MTi-680G providing RTK-capable GNSS/INS positioning with centimeter-level accuracy when RTK correction is available. For this research, we leverage the unit's integrated tri-axial accelerometer and gyroscope, which provide:

\begin{itemize}
    \item \textbf{Accelerometer:} $\pm$16~g range, 100~Hz output rate, noise density $<$100~$\mu$g/$\sqrt{\text{Hz}}$
    \item \textbf{Gyroscope:} $\pm$2000~$^\circ$/s range, 100~Hz output rate
    \item \textbf{Output:} Free acceleration in ENU (East-North-Up) frame, calibrated angular velocity in body frame
\end{itemize}

The sensor is mounted rigidly to the vehicle chassis near the center of mass. The General\_RTK filter profile is used, which relies on GNSS and inertial fusion without magnetometer input---appropriate for operation near metal structures.

\begin{figure}[htbp]
\centering
% TODO: Add vehicle photo with sensor locations marked
\includegraphics[width=0.9\columnwidth]{figures/vehicle_photo.jpg}
\caption{The DRIVE platform with Xsens MTi-680G IMU location indicated. The sensor is mounted rigidly to the chassis to capture vehicle body vibrations.}
\label{fig:vehicle}
\end{figure}

\subsection{Software Architecture}

The vehicle's software follows a modular sense-plan-control-act pipeline implemented in Python. The existing \texttt{XsensReceiver} class provides thread-safe access to IMU data at 100~Hz, including free acceleration in ENU frame and calibrated gyroscope readings in body frame.

For this research, we developed three additional modules:
\begin{itemize}
    \item \texttt{VibrationLogger}: High-rate data recording with GPS timestamps and surface type labels
    \item \texttt{VibrationFeatures}: Feature extraction for time and frequency domains
    \item \texttt{RoadClassifier}: Machine learning inference with multiple algorithm support
\end{itemize}

%==============================================================================
\section{Methodology}

\subsection{Data Acquisition}

Vibration data is acquired from the Xsens MTi-680G at 100~Hz. While the sensor outputs free acceleration in the ENU (East-North-Up) frame, we primarily use the vertical (Up) component for road surface classification, as vertical accelerations most directly reflect road roughness. The horizontal components (East, North) provide supplementary information about vehicle pitch and roll induced by surface irregularities.

Data is segmented into overlapping windows for analysis:
\begin{itemize}
    \item Window size: 1.0 second (100 samples)
    \item Overlap: 50\% (hop size 0.5 seconds)
    \item Resulting classification rate: 2 predictions per second
\end{itemize}

Each window is labeled with the road surface type based on GPS location and pre-mapped surface regions on campus.

\subsection{Feature Extraction}

We extract features from both time and frequency domains to capture complementary aspects of road surface characteristics.

\subsubsection{Time-Domain Features}

For each 1-second window of vertical acceleration $a[n]$ with $N=100$ samples:

\textbf{RMS Acceleration:}
\begin{equation}
a_{\text{RMS}} = \sqrt{\frac{1}{N}\sum_{n=0}^{N-1} a[n]^2}
\end{equation}

\textbf{Peak-to-Peak Amplitude:}
\begin{equation}
a_{\text{pp}} = \max(a[n]) - \min(a[n])
\end{equation}

\textbf{Crest Factor:}
\begin{equation}
\text{CF} = \frac{\max(|a[n]|)}{a_{\text{RMS}}}
\end{equation}

\textbf{Kurtosis:}
\begin{equation}
\kappa = \frac{\frac{1}{N}\sum_{n=0}^{N-1}(a[n] - \bar{a})^4}{\left(\frac{1}{N}\sum_{n=0}^{N-1}(a[n] - \bar{a})^2\right)^2}
\end{equation}

\textbf{Skewness:}
\begin{equation}
\gamma = \frac{\frac{1}{N}\sum_{n=0}^{N-1}(a[n] - \bar{a})^3}{\left(\frac{1}{N}\sum_{n=0}^{N-1}(a[n] - \bar{a})^2\right)^{3/2}}
\end{equation}

\textbf{Zero-Crossing Rate:}
\begin{equation}
\text{ZCR} = \frac{1}{N-1}\sum_{n=1}^{N-1} \mathbf{1}[a[n] \cdot a[n-1] < 0]
\end{equation}

\subsubsection{Frequency-Domain Features}

We compute the magnitude spectrum using FFT:
\begin{equation}
A[k] = \left|\sum_{n=0}^{N-1} a[n] \cdot e^{-j2\pi kn/N}\right|
\end{equation}

\textbf{Spectral Centroid:}
\begin{equation}
f_c = \frac{\sum_{k=0}^{N/2} f[k] \cdot A[k]}{\sum_{k=0}^{N/2} A[k]}
\end{equation}

\textbf{Band Power Ratios:} We divide the spectrum into three bands:
\begin{itemize}
    \item Low: 0--5~Hz (long-wavelength roughness)
    \item Mid: 5--15~Hz (medium-wavelength roughness)
    \item High: 15--50~Hz (fine texture, discrete events)
\end{itemize}

Power in each band is computed as the sum of squared magnitudes, and ratios between bands serve as features.

\textbf{Dominant Frequency:}
\begin{equation}
f_{\text{dom}} = \arg\max_k A[k], \quad k \in [1, N/2]
\end{equation}

\subsection{Classification Models}

We evaluate both traditional machine learning and deep learning approaches.

\subsubsection{Traditional Machine Learning}

\textbf{Random Forest:} An ensemble of decision trees provides robust classification with built-in feature importance metrics. We use 100 trees with maximum depth of 10.

\textbf{Support Vector Machine:} SVM with RBF kernel captures nonlinear decision boundaries. Hyperparameters $C$ and $\gamma$ are tuned via grid search.

\textbf{Gradient Boosting:} Sequential ensemble method that corrects errors of previous estimators, often achieving high accuracy on tabular data.

\subsubsection{Deep Learning}

\textbf{1D-CNN:} Processes raw accelerometer windows directly, learning features automatically:
\begin{lstlisting}
Input: (batch, 100, 3)  # 3 axes
Conv1D(32, kernel=5) -> ReLU -> MaxPool
Conv1D(64, kernel=5) -> ReLU -> MaxPool
Flatten -> Dense(64) -> ReLU
Dense(num_classes) -> Softmax
\end{lstlisting}

\textbf{CNN-LSTM:} Adds temporal modeling to capture dependencies across consecutive windows:
\begin{lstlisting}
Input: (batch, 100, 3)
Conv1D(32, kernel=5) -> ReLU -> MaxPool
Conv1D(64, kernel=5) -> ReLU -> MaxPool
LSTM(64) -> Dropout(0.3)
Dense(32) -> ReLU -> Dropout(0.3)
Dense(num_classes) -> Softmax
\end{lstlisting}

\subsection{Training Protocol}

Data is split into training (70\%), validation (15\%), and test (15\%) sets, stratified by surface class. We employ 5-fold cross-validation for hyperparameter selection. Models are trained to minimize categorical cross-entropy loss using Adam optimizer with learning rate $10^{-3}$.

To address class imbalance, we apply class weights inversely proportional to class frequency. Data augmentation includes adding Gaussian noise ($\sigma = 0.01$~g) and time shifting ($\pm$5 samples).

%==============================================================================
\section{Experimental Setup}

\subsection{Data Collection Routes}

Data was collected on the Cal Poly Pomona campus across five road surface types:

\begin{enumerate}
    \item \textbf{Smooth Asphalt:} Recently paved roads with minimal defects
    \item \textbf{Rough Asphalt:} Aged pavement with visible weathering and minor cracking
    \item \textbf{Speed Bumps:} Standard parking lot speed reduction devices
    \item \textbf{Potholes/Patches:} Sections with visible repairs or surface damage
    \item \textbf{Concrete:} Pedestrian pathways with expansion joints
\end{enumerate}

% TODO: Add campus map figure showing data collection routes
\begin{figure}[htbp]
\centering
% \includegraphics[width=0.9\columnwidth]{figures/campus_routes.png}
\caption{Data collection routes on Cal Poly Pomona campus. Different colors indicate surface types. [TODO: Add figure]}
\label{fig:routes}
\end{figure}

\subsection{Collection Protocol}

For each surface type, data was collected at three speeds: 5, 10, and 15~mph (8, 16, and 24~km/h). Each route was traversed at least three times per speed, resulting in minimum 9 passes per surface type.

Collection sessions occurred on dry days with minimal wind to reduce environmental variability. The vehicle was operated in autonomous mode following pre-defined waypoints to ensure consistent trajectories across passes.

\begin{table}[htbp]
\caption{Dataset Statistics}
\label{tab:dataset}
\centering
\begin{tabular}{@{}lccc@{}}
\toprule
\textbf{Surface Type} & \textbf{Duration (s)} & \textbf{Windows} & \textbf{\% of Total} \\
\midrule
Smooth Asphalt & [TODO] & [TODO] & [TODO] \\
Rough Asphalt & [TODO] & [TODO] & [TODO] \\
Speed Bumps & [TODO] & [TODO] & [TODO] \\
Potholes/Patches & [TODO] & [TODO] & [TODO] \\
Concrete & [TODO] & [TODO] & [TODO] \\
\midrule
\textbf{Total} & [TODO] & [TODO] & 100\% \\
\bottomrule
\end{tabular}
\end{table}

%==============================================================================
\section{Results}

\subsection{Classification Performance}

% TODO: Fill in results after experiments

\begin{table}[htbp]
\caption{Classification Accuracy by Model}
\label{tab:accuracy}
\centering
\begin{tabular}{@{}lcccc@{}}
\toprule
\textbf{Model} & \textbf{Accuracy} & \textbf{Precision} & \textbf{Recall} & \textbf{F1} \\
\midrule
Random Forest & [TODO] & [TODO] & [TODO] & [TODO] \\
SVM (RBF) & [TODO] & [TODO] & [TODO] & [TODO] \\
Gradient Boosting & [TODO] & [TODO] & [TODO] & [TODO] \\
1D-CNN & [TODO] & [TODO] & [TODO] & [TODO] \\
CNN-LSTM & [TODO] & [TODO] & [TODO] & [TODO] \\
\bottomrule
\end{tabular}
\end{table}

\subsection{Confusion Matrix Analysis}

% TODO: Add confusion matrix figure
\begin{figure}[htbp]
\centering
% \includegraphics[width=0.9\columnwidth]{figures/confusion_matrix.png}
\caption{Confusion matrix for CNN-LSTM classifier. [TODO: Add figure]}
\label{fig:confusion}
\end{figure}

\subsection{Feature Importance}

For the Random Forest model, we analyze feature importance to understand which measurements most strongly distinguish surface types.

% TODO: Add feature importance figure
\begin{figure}[htbp]
\centering
% \includegraphics[width=0.9\columnwidth]{figures/feature_importance.png}
\caption{Feature importance rankings from Random Forest classifier. [TODO: Add figure]}
\label{fig:features}
\end{figure}

\subsection{Real-Time Performance}

Inference latency is critical for autonomous vehicle applications. We benchmark each model on the vehicle's Jetson Orin AGX compute platform:

\begin{table}[htbp]
\caption{Inference Latency (ms)}
\label{tab:latency}
\centering
\begin{tabular}{@{}lcc@{}}
\toprule
\textbf{Model} & \textbf{CPU} & \textbf{GPU} \\
\midrule
Random Forest & [TODO] & N/A \\
SVM (RBF) & [TODO] & N/A \\
1D-CNN & [TODO] & [TODO] \\
CNN-LSTM & [TODO] & [TODO] \\
\bottomrule
\end{tabular}
\end{table}

\subsection{Speed Dependency Analysis}

Vehicle speed affects vibration characteristics. We analyze classification performance as a function of speed:

% TODO: Add speed dependency figure

\subsection{Comparison with Literature}

% TODO: Add comparison table with other published results

%==============================================================================
\section{Educational Integration}

\subsection{Curriculum Context}

This research was conducted within the broader DRIVE (Development Research Infrastructure for Vehicle Education) program at Cal Poly Pomona. The program aims to provide undergraduate students hands-on experience with autonomous vehicle development at a fraction of commercial platform costs.

The road surface classification project specifically supports learning objectives in:
\begin{itemize}
    \item \textbf{Signal Processing:} Students implement windowing, FFT computation, and feature extraction from real sensor data
    \item \textbf{Machine Learning:} Training, validation, and deployment of classification models
    \item \textbf{Embedded Systems:} Real-time inference on resource-constrained platforms
    \item \textbf{Systems Integration:} Incorporating new perception capabilities into existing autonomous vehicle software
\end{itemize}

\subsection{Student Engagement}

% TODO: Add student engagement metrics if available

The modular software architecture enables students to engage at multiple levels:
\begin{itemize}
    \item \textbf{Beginner:} Collect labeled data, visualize features, evaluate pre-trained models
    \item \textbf{Intermediate:} Implement new feature extraction methods, tune hyperparameters
    \item \textbf{Advanced:} Design novel model architectures, optimize for real-time deployment
\end{itemize}

\subsection{Broader Impact}

This project demonstrates that meaningful research contributions can emerge from educational platforms. By publishing open-source code and documented methodology, we enable other institutions to replicate and extend this work.

%==============================================================================
\section{Conclusion}

This paper demonstrated that navigation-grade IMUs, already present on autonomous vehicles, can effectively classify road surface conditions without additional hardware. Using the Xsens MTi-680G's 100~Hz tri-axial accelerometer, we achieved [TODO]\% classification accuracy across five surface types using a CNN-LSTM architecture with inference latency under 50~ms.

The dual-purpose use of navigation IMUs for both localization and vibration sensing reduces system complexity while enabling environmental perception capabilities relevant to ride comfort, energy efficiency, and safety. Our results suggest that autonomous vehicles can develop real-time awareness of road conditions using existing sensor suites.

From an educational perspective, this research demonstrates how undergraduate students can contribute to meaningful research while developing skills in signal processing, machine learning, and embedded systems. The DRIVE platform provides an accessible testbed for institutions seeking to establish autonomous vehicle education programs.

\subsection{Limitations}

Several limitations should be acknowledged:
\begin{itemize}
    \item Data collected on a single campus may not generalize to all road types
    \item The electric UTV has different suspension characteristics than passenger vehicles
    \item Weather conditions (wet pavement) were not evaluated
    \item Single IMU placement; multiple locations might improve accuracy
\end{itemize}

\subsection{Future Work}

Future directions include:
\begin{itemize}
    \item Expanding the dataset to include more surface types and environmental conditions
    \item Investigating transfer learning across different vehicle platforms
    \item Integrating surface classification with path planning for route optimization
    \item Combining vibration analysis with acoustic noise mapping for comprehensive environmental sensing
\end{itemize}

%==============================================================================
\section*{Acknowledgment}

This work was supported by the Ganpat and Manju Center for International Collaboration and Innovation at Cal Poly Pomona, established through the generous contribution of Dr. Ganpat Patel and Manju Patel. The DRIVE platform was developed as part of the iCARE-M\&S program. The authors thank the members of the Autonomous Vehicle Laboratory for their contributions to platform development.

%==============================================================================
\bibliographystyle{IEEEtran}
\bibliography{references}

\balance

\end{document}
