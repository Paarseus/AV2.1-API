\documentclass[11pt]{article}
\usepackage[margin=1in]{geometry}
\usepackage{booktabs}
\usepackage{enumitem}
\usepackage{xcolor}
\usepackage{fancybox}
\usepackage{tcolorbox}
\usepackage{listings}
\usepackage{hyperref}

\definecolor{codegreen}{rgb}{0,0.6,0}
\definecolor{codegray}{rgb}{0.5,0.5,0.5}
\definecolor{codepurple}{rgb}{0.58,0,0.82}
\definecolor{backcolour}{rgb}{0.95,0.95,0.92}

\lstdefinestyle{mystyle}{
    backgroundcolor=\color{backcolour},
    commentstyle=\color{codegreen},
    keywordstyle=\color{codepurple},
    numberstyle=\tiny\color{codegray},
    basicstyle=\ttfamily\small,
    breakatwhitespace=false,
    breaklines=true,
    keepspaces=true,
    showspaces=false,
    showstringspaces=false,
    showtabs=false,
    tabsize=2,
    frame=single
}
\lstset{style=mystyle}

\title{\textbf{Lab Demo Plan}\\[0.5em]\large Ulsan Student Visit --- January 22, 2026}
\author{Autonomous Vehicle Laboratory\\Cal Poly Pomona}
\date{}

\begin{document}
\maketitle

\section*{Visit Information}

\begin{table}[h]
\centering
\begin{tabular}{lll}
\toprule
\textbf{Group} & \textbf{Time} & \textbf{Lead} \\
\midrule
Group B (30 students) & 1:00pm -- 2:20pm & Yvonne Chen \\
Group A (31 students) & 2:30pm -- 3:50pm & James Wu \\
\bottomrule
\end{tabular}
\end{table}

\noindent\textbf{Location:} Electric Vehicle Lab, Building 9, Room 101\\
\textbf{Duration:} 80 minutes per group

\section*{Hardware Status}
\begin{itemize}[noitemsep]
    \item Vehicle drives manually
    \item GPS/IMU working (individually)
    \item LIDAR working (individually)
    \item Camera working
    \item Sim chair available for teleoperation
\end{itemize}

\hrule
\vspace{1em}

\section{Schedule Overview (80 min)}

\begin{table}[h]
\centering
\begin{tabular}{clc}
\toprule
\textbf{Time} & \textbf{Activity} & \textbf{Duration} \\
\midrule
0:00 & Welcome \& Research Overview & 12 min \\
0:12 & Algorithm Visualization Demos & 20 min \\
0:32 & Live Sensor Demos & 15 min \\
0:47 & Hands-On Vehicle Interaction & 25 min \\
1:12 & Q\&A and Discussion & 8 min \\
\bottomrule
\end{tabular}
\end{table}

\newpage

\section{Welcome \& Research Overview (12 min)}

\subsection*{Introduction (5 min)}
\begin{itemize}[noitemsep]
    \item Welcome to the Autonomous Vehicle Laboratory
    \item Our research: Full-scale AV testbed for GPS-guided autonomous navigation
    \item Introduce the \textbf{Sense-Plan-Control-Act} pipeline
\end{itemize}

\subsection*{Architecture Diagram}
\begin{lstlisting}
Sensors -> Perception -> Planning -> Control -> Actuators
   |           |            |           |           |
GPS/IMU    occupancy    navigator  pure_pursuit  vehicle
LIDAR      costmap      DWA        PID           (CAN bus)
camera     BEV          path plan  steering
\end{lstlisting}

\subsection*{Research Challenges (7 min)}
Engage students with questions:
\begin{enumerate}[noitemsep]
    \item ``What sensors would you use to localize a car?''
    \item ``Why can't we just use GPS?'' (tunnels, urban canyons, accuracy)
    \item ``How do you avoid obstacles in real-time?''
    \item ``What makes autonomous driving hard?''
\end{enumerate}

\section{Algorithm Visualization Demos (20 min)}

\subsection{Demo A: Dynamic Window Approach (8 min)}
\begin{lstlisting}[language=bash]
python examples/ackermann_dwa_synthetic.py
\end{lstlisting}

\textbf{What students see:} Animated vehicle navigating around obstacles

\textbf{Key points:}
\begin{itemize}[noitemsep]
    \item DWA samples hundreds of possible trajectories
    \item Ackermann steering = car-like constraints
    \item Cost function: reach goal vs avoid obstacles vs prefer speed
    \item Real-time: must compute in $<$50ms
\end{itemize}

\textbf{Engagement:} ``Predict which path it will choose!''

\subsection{Demo B: Occupancy Grid \& Costmap (7 min)}
\begin{lstlisting}[language=bash]
python examples/costmap_synthetic.py
\end{lstlisting}

\textbf{What students see:} Side-by-side probability grid vs inflated costmap

\textbf{Key points:}
\begin{itemize}[noitemsep]
    \item Occupancy grid: each cell = probability of obstacle (0.0 to 1.0)
    \item Bayesian updates: sensor readings update beliefs
    \item Inflation: add safety margin = robot radius + buffer
\end{itemize}

\subsection{Demo C: 3D Sensor Fusion (5 min)}
\begin{lstlisting}[language=bash]
python tests/test_bev_lidar_synthetic.py
\end{lstlisting}

\textbf{What students see:} Interactive 3D view -- BEV camera + LIDAR points

\textbf{Hands-on:} Let a student rotate/zoom the 3D view

\section{Live Sensor Demos (15 min)}

\subsection{Demo D: LIDAR Point Cloud (7 min)}
\begin{lstlisting}[language=bash]
python tests/lidar/01_raw_pointcloud.py
\end{lstlisting}

\textbf{What students see:} Real-time 3D point cloud from Velodyne LIDAR

\textbf{Key points:}
\begin{itemize}[noitemsep]
    \item Velodyne spins 360° at 10--20 Hz
    \item Each point = (x, y, z, intensity)
    \item \textbf{Demo:} Walk in front of LIDAR, watch yourself appear as points
\end{itemize}

\subsection{Demo E: GPS/IMU Tracking (5 min)}
\begin{lstlisting}[language=bash]
python tests/example_mti_receive_data.py
\end{lstlisting}

\textbf{What students see:} Real-time position, heading, velocity from Xsens

\textbf{Key points:}
\begin{itemize}[noitemsep]
    \item RTK GPS: $\sim$2cm accuracy (vs $\sim$3m for phone GPS)
    \item IMU: 100Hz attitude updates
\end{itemize}

\subsection{Demo F: Camera Perception (3 min)}
\begin{lstlisting}[language=bash]
python examples/perception_demo.py --camera rgb --camera-id 0 --visualizer 2d
\end{lstlisting}

\textbf{What students see:} Live camera with BEV transformation

\section{Hands-On Vehicle Interaction (25 min)}

\subsection{Demo G: Sim Chair Teleoperation (15 min)}

\textbf{Student participation:}
\begin{itemize}[noitemsep]
    \item Let 3--4 students try driving via sim chair
    \item Set up simple course with cones
    \item ``Who can navigate the course smoothest?''
\end{itemize}

\subsection{Demo H: WebUI Phone Control (10 min)}
\begin{lstlisting}[language=bash]
cd webui && python3 server.py
\end{lstlisting}

\textbf{Features to show:}
\begin{itemize}[noitemsep]
    \item Touch joystick (throttle + steering gauges)
    \item Voice commands: ``go'', ``stop'', ``left'', ``right''
    \item Emergency E-STOP button
\end{itemize}

\textbf{Student participation:} Let 2--3 students try voice commands

\section{Q\&A and Discussion (8 min)}

\begin{table}[h]
\centering
\begin{tabular}{p{5cm}p{8cm}}
\toprule
\textbf{Question} & \textbf{Answer} \\
\midrule
``What programming languages?'' & Python for algorithms, C++ for Teensy firmware \\
``How accurate is GPS?'' & RTK = $\sim$2cm, Standard = $\sim$3m \\
``How fast can it go?'' & Max 2.5 m/s, limited to 1 m/s for safety \\
``Hardest part of AV?'' & Handling uncertainty, edge cases, real-time safety \\
``Can it work in rain?'' & LIDAR yes, camera struggles, GPS works \\
``Career paths in AV?'' & Software, perception, controls, systems engineering \\
\bottomrule
\end{tabular}
\end{table}

\newpage

\section*{Pre-Visit Checklist}

\subsection*{Morning Setup}
\begin{itemize}
    \item[$\square$] Charge vehicle batteries
    \item[$\square$] Power on Velodyne LIDAR, verify UDP streaming
    \item[$\square$] Power on Xsens GPS/IMU, check fix quality
    \item[$\square$] Connect sim chair, test teleoperation
    \item[$\square$] Test WebUI on phone
    \item[$\square$] Run each synthetic demo once
    \item[$\square$] Set up projector and test display
    \item[$\square$] Clear 5m x 5m space for vehicle movement
    \item[$\square$] Position traffic cones for driving course
    \item[$\square$] E-STOP button accessible and tested
\end{itemize}

\subsection*{Software Verification}
\begin{lstlisting}[language=bash]
# Run verification script
python lab_demo/verify_demos.py

# Or test manually:
python examples/ackermann_dwa_synthetic.py
python examples/costmap_synthetic.py
python tests/test_bev_lidar_synthetic.py
python tests/lidar/00_basic_test.py
python tests/example_mti_receive_data.py
\end{lstlisting}

\section*{Backup Plans}

\textbf{If LIDAR fails:} Skip Demo D, extend algorithm demos

\textbf{If vehicle won't drive:} Focus on sim chair as ``the interface''

\textbf{If projector fails:} Gather students around laptop screens

\textbf{If running behind:} Cut Demo F (3 min), shorten Q\&A to 5 min

\section*{Quick Command Reference}

\begin{lstlisting}[language=bash]
# Algorithm Demos (No Hardware)
python examples/ackermann_dwa_synthetic.py    # Demo A
python examples/costmap_synthetic.py          # Demo B
python tests/test_bev_lidar_synthetic.py      # Demo C

# Live Sensor Demos (Hardware Required)
python tests/lidar/01_raw_pointcloud.py       # Demo D
python tests/example_mti_receive_data.py      # Demo E
python examples/perception_demo.py --camera rgb --camera-id 0  # Demo F

# Vehicle Control
cd webui && python3 server.py                 # Demo H
\end{lstlisting}

\end{document}
